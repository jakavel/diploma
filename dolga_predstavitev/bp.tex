\documentclass{beamer}
\usepackage[utf8]{inputenc}
\usepackage[slovene]{babel}
\usepackage{amsmath}
\usepackage{mathtools}
\usepackage{multicol}
\usepackage{mathtools}
\renewcommand{\vec}{\underline}
\usetheme{Madrid}
\setbeamertemplate{navigation symbols}{}

\title[Razširjanje zaupanja]{Razširjanje zaupanja za lokalizacijo senzorskih omrežji}
\author[Jaka Velkaverh]{Jaka Velkaverh\\Mentorja: Sergio Cabello, Tomaž Javornik}
\date{22.3.2024}

\theoremstyle{definition}
\newtheorem{definicija}{Definicija}

\begin{document}

	\frame{\titlepage}

	\begin{frame}
		\frametitle{Cilj}
		\begin{block}{Lokalizacija senzorskega omrežja}
			Omrežje si predstavljamo kot utežen graf, kjer so naprave vozlišča.
			Dve vozlišči sta povezani, če sta pripadajoči napravi dovolj blizu,
			da lahko izmerita razdaljo. Utež povezav je izmerjena razdalja,
			kjer pri merjenju pride do napak.

			Vemo lokacije od nekaterih naprav, tem rečemo sidra. Izračunati
			želimo približke za lokacije ostalih, tem rečemo agenti.
		\end{block}
		Predpostavimo, da je omrežje naprav povezano.
	\end{frame}

	\begin{frame}
		\frametitle{Oznake}
		Oznake
		\begin{itemize}
			\item Vozlišča označujemo z črkami $u, v, s \in V$.
			\item $x_u \in \mathbb{R}^n\;;\;n=2, 3$ lokacija naprave u.
			\item Za $A \coloneqq \left\{v_1, v_2, \ldots, v_k\right\} \subseteq V$ je
				$\vec{x}_A = \left(x_{v_1}, x_{v_2}, \ldots, x_{v_k}\right)$
			\item Za $uv \in E$ je $d_{u,v}$ izmerjena razdalja.
		\end{itemize}
	\end{frame}

	\begin{frame}
		$$E\left(\vec{x}_V\right) \coloneqq \sum_{uv \in E}\left(\left|x_u - x_v\right|-d_{u,v}\right)^2$$
		\pause
		$$P\left(\vec{x}_V\right) = exp\left[-\sum_{uv \in E}\left(\left|x_u - x_v\right|-d_{u,v}\right)^2\right] =
		\prod_{uv \in E}e^{-\left(\left|x_u - x_v\right|-d_{u,v}\right)^2}$$
		$$
		= \prod_{uv \in E}\psi_{u,v}\left(x_u,x_v\right)
		$$
		\pause
		Omejimo na $S \subseteq \mathbb{R}^{n\cdot \left|V\right|}$ s končnim volumnom.
		$$\tilde{P}\left(\vec{x}_V\right) = \left(\int_S P\left(\vec{z}_V\right)d\vec{z}_V\right)^{-1}
		P\left(\vec{x}_V\right)$$
		\pause
		$\vec{X}_V$ zvezno porazdeljen slučajni vektor z gosto $\tilde{P}$.
	\end{frame}

	\begin{frame}
		\frametitle{Markovska polja}
		\begin{block}{Pogojna neodvisnost}
			Nakjučna vektorja $\vec{X}, \vec{Y}$ sta pogojno
			neodvisna ob $\vec{Z}$, če velja:
			$$f_{\vec{X}, \vec{Y}|\vec{Z}}\left(\vec{x}, \vec{y}\ |\ \vec{Z} = \vec{z}\right) =
			f_{\vec{X}|\vec{Z}}\left(\vec{x}\ |\ \vec{Z} = \vec{z}\right) \
			f_{\vec{Y}|\vec{Z}}\left(\vec{y}\ |\ \vec{Z} = \vec{z}\right)
			$$
		\end{block}
		\pause
		\begin{definicija}
			Graf $G = \left(V, E\right)$, in naključni slučajni vektorji
			$\vec{X}_v$, indeksirani z $v \in V$ tvorijo Markovsko polje,
			če velja Markova lastnost po parih:

			Za vsaka nesosedna $u, v \in V$ sta $\vec{X}_u$, $\vec{X}_v$
			pogojno neodvisna ob vseh ostalih
			$\vec{X}_{V\backslash\left\{u, v\right\}}$
		\end{definicija}
	\end{frame}

	\begin{frame}
		\frametitle{$\vec{X}_V$ je Markovsko polje}
		Za nesosednja $u, v \in V$ je
		$$
		f_{x_u,x_v|\vec{X}_{V\backslash \left\{u,v\right\}}}
			\left(x_u,x_y | \vec{X}_{V\backslash \left\{u,v\right\}} = \vec{y}_{V\backslash \left\{u,v\right\}}\right) \propto
		$$
		\pause
		$$
		\propto \prod_{su \in E}\psi_{s,v}\left(y_s,x_u\right)
			\prod_{sv \in E}\psi_{s,v}\left(y_s,x_v\right)
			\prod_{\substack{\left\{s,t\right\} \in E \\ \left\{s,t\right\}\cap \left\{u,v\right\} = \emptyset}}\psi_{s,v}\left(y_s,y_t\right)
			\propto
		$$
		\pause
		$$
		\propto g\left(x_u\right) h\left(x_v\right) \Rightarrow X_u, X_v \text{ pogojno neodvisna ob } \vec{X}_{V\backslash \left\{u,v\right\}}
		$$
	\end{frame}

\end{document}

