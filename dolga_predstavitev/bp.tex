\documentclass{beamer}
\usepackage[utf8]{inputenc}
\usepackage[slovene]{babel}
\usepackage{amsmath}
\usepackage{mathtools}
\usepackage{multicol}
\usepackage{mathtools}
\renewcommand{\vec}{\underline}
\usetheme{Madrid}
\setbeamertemplate{navigation symbols}{}

\title[Razširjanje zaupanja]{Razširjanje zaupanja za lokalizacijo senzorskih omrežji}
\author[Jaka Velkaverh]{Jaka Velkaverh\\Mentorja: Sergio Cabello, Tomaž Javornik}
\date{22.3.2024}

\theoremstyle{definition}
\newtheorem{definicija}{Definicija}

\begin{document}

	\frame{\titlepage}

	\begin{frame}
		\frametitle{Cilj}
		\begin{block}{Lokalizacija senzorskega omrežja}
			Omrežje si predstavljamo kot utežen graf, kjer so naprave vozlišča.
			Dve vozlišči sta povezani, če sta pripadajoči napravi dovolj blizu,
			da lahko izmerita razdaljo. Utež povezav je izmerjena razdalja,
			kjer pri merjenju pride do napak.

			Vemo lokacije od nekaterih naprav, tem rečemo sidra. Izračunati
			želimo približke za lokacije ostalih, tem rečemo agenti.
		\end{block}
		Predpostavimo, da je omrežje naprav povezano.
	\end{frame}

	\begin{frame}
		\frametitle{Oznake}
		Oznake
		\begin{itemize}
			\item Vozlišča označujemo z črkami $u, v, s \in V$.
			\item $x_u \in \mathbb{R}^n\;;\;n=2, 3$ lokacija naprave u.
			\item Za $A \coloneqq \left\{v_1, v_2, \ldots, v_k\right\} \subseteq V$ je
				$\vec{x}_A = \left(x_{v_1}, x_{v_2}, \ldots, x_{v_k}\right)$
			\item Za $uv \in E$ je $d_{u,v}$ izmerjena razdalja.
		\end{itemize}
	\end{frame}

	\begin{frame}
		$$E\left(\vec{x}_V\right) \coloneqq \sum_{uv \in E}\left(\left|x_u - x_v\right|-d_{u,v}\right)^2$$
		\pause
		$$P\left(\vec{x}_V\right) = exp\left[-\sum_{uv \in E}\left(\left|x_u - x_v\right|-d_{u,v}\right)^2\right] =
		\prod_{uv \in E}e^{-\left(\left|x_u - x_v\right|-d_{u,v}\right)^2}$$
		\pause
		Omejimo na $S \subseteq \mathbb{R}^{n\cdot \left|V\right|}$ s končnim volumnom.
		$$\tilde{P}\left(\vec{x}_V\right) = \left(\int_S P\left(\vec{z}_V\right)d\vec{z}_V\right)^{-1}
		P\left(\vec{x}_V\right)$$
		\pause
		$\vec{X}_V$ zvezno porazdeljen slučajni vektor z gosto $\tilde{P}$.

	\end{frame}

\end{document}

