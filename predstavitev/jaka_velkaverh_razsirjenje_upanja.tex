\documentclass{beamer}
\usepackage[utf8]{inputenc}
\usepackage[slovene]{babel}
\usepackage{amsmath}
\usepackage{mathtools}
\usepackage{multicol}
\usetheme{Madrid}
\setbeamertemplate{navigation symbols}{}

\title[Razširjanje zaupanja]{Razširjanje zaupanja za lokalizacijo senzorskih omrežji}
\author{Jaka Velkaverh}
\date{18.12.2023}

\theoremstyle{definition}
\newtheorem{definicija}{Definicija}

\begin{document}

	\frame{\titlepage}

	\begin{frame}
		\frametitle{Opis problema}
		\begin{block}{Lokalizacija senzorskega omrežja}
			Omrežje si predstavljamo kot utežen graf, kjer so naprave vozlišča.
			Dve vozlišči sta povezani, če sta pripadajoči napravi dovolj blizu,
			da lahko izmerita razdaljo. Utež povezav je izmerjena razdalja,
			kjer pri merjenju pride do napak.

			Vemo lokacije od nekaterih naprav, tem rečemo sidra. Izračunati
			želimo približke za lokacije ostalih, tem rečemo agenti.
		\end{block}
	\end{frame}
	\begin{frame}
		\frametitle{Verjetnostni pogled}
		Oznake
		\begin{itemize}
			\item Vozlišča označujemo z črkami $u, v, s$.
			\item Napaka pri merjenju $\nu_{u, v} \sim N\left(0, \sigma^2\right)$.
			\item $\underline{X}_u \in K\left(0,R\right) \subseteq \mathbb{R}^n$ naključni vektor lokacije naprave.
			\item $\underline{X}_A$ naključni vektor lokacije množice naprav $A$.
		\end{itemize}
	\end{frame}

	\begin{frame}
		\frametitle{Markovska polja}
		\begin{block}{Pogojna neodvisnost}
			Nakjučna vektorja $\underline{X}, \underline{Y}$ sta pogojno
			neodvisna ob $\underline{Z}$, če velja:
			$$f_{\underline{X}, \underline{Y}}\left(\underline{x}, \underline{y}\ |\ \underline{Z} = \underline{z}\right) =
			  f_{\underline{X}}\left(\underline{x}\ |\ \underline{Z} = \underline{z}\right) \
			  f_{\underline{Y}}\left(\underline{y}\ |\ \underline{Z} = \underline{z}\right)
			$$
		\end{block}
		\pause
		\begin{definicija}
			Graf $G = \left(V, E\right)$, in naključni slučajni vektorji
			$\underline{X}_v$, indeksirani z $v \in V$ tvorijo Markovsko polje,
			če velja Markova lastnost po parih:

			Za vsaka nesosedna $u, v \in V$ sta $\underline{X}_u$, $\underline{X}_v$
			pogojno neodvisna ob vseh ostalih
			$\underline{X}_{V\backslash\left\{u, v\right\}}$
		\end{definicija}
	\end{frame}

	\begin{frame}
		\begin{block}{Hammersley–Cliffordov izrek}
			Naj bo $G = \left(V, E\right)$ in $\underline{X}_v$ kot prej in dodatno
			$f_{\underline{X}_V}\left(\underline{x}_V\right) > 0\quad \forall \underline{x}_V$.
			Tedaj $G$ in $\underline{X}_V$ tvorita Markovsko polje natanko tedaj,
			ko se skupna gostota faktorizira po klikah grafa $G$:
			$$f_{\underline{X}_V}\left(\underline{x}_V\right) \propto
				\prod_{C\;\text{klika}\;G}\psi_C\left(\underline{x}_C\right)$$
		\end{block}
		\pause
		Torej če definiramo skupno gostoto kot produkt potencialov, bodo
		pripadajoče slučajne spremenljivke tvorile Markovsko polje.
	\end{frame}

	\begin{frame}
		Za $\left\{u,v\right\} \in E$ in $d_{u,v}$ izmerjena razdalja definiramo
		$$
		\psi_{u,v}\left(\underline{x}_u,\underline{x}_v\right) =
		f_{\nu_{u, v}}\left(\left|\underline{x}_u-\underline{x}_v\right| - d_{u,v}\right) =
		\frac{1}{\sigma\sqrt{2\pi}} \exp\left[-\frac{\left(\left|\underline{x}_u-\underline{x}_v\right| - d_{u,v}\right)^2}{2 \sigma^2}\right]
		$$
		Za klike $C$, ki niso velikosti $2$
		$$
		\psi_C\left(\underline{x}_C\right)=1
		$$
		Če imamo predhodne informacije o napravah jih lahko vstavimo v enojne potenciale.
	\end{frame}

	\begin{frame}
		\begin{figure}[t]
			\includegraphics[width=7cm]{"pairwise_potential.png"}
			\caption{Primer potenciala parov, če enega od pozicij fiksiramo. Tu je izmerjena razdalja $10$, $\sigma$ pa $0{,}3$.}
			\centering
		\end{figure}
	\end{frame}

	\begin{frame}
		Naj bodo $V_a$ agenti, $V_s$ pa sidra.
		Z $\underline{y}_{V_s}$ označimo (fiksne) znane pozicije sider.
		Posteriorna gostota bo torej
		$$
		f_{V_a | V_s}\left(\underline{x}_{V_a} | V_s = \underline{y}_{V_s}\right) \propto
		\prod_{\substack{\left\{u,v\right\} \in E \\ u,v \text{ agenta}}}\psi_{u,v}\left(\underline{x}_u,\underline{x}_v\right)
		\prod_{\substack{\left\{u,v\right\} \in E \\ u\text{ agent}, v \text{ sidro}}}\psi_{u,v}\left(\underline{x}_u,\underline{y}_v\right)
		$$
		\pause
		Za $u \in V_a$ želimo izračunati robno gostoto
		$$
		f_{X_u | V_s}\left(\underline{x}_u | V_s = \underline{y}_{V_s}\right) =
		\int_{V_a\backslash X_u} f_{V_a | V_s}\left(\underline{x}_{V_a\backslash X_u} | V_s = \underline{y}_{V_s}\right)
		d\underline{x}_{V_a\backslash X_u}
		$$
	\end{frame}

	\begin{frame}
		Poglejmo si primer, ko je graf drevo.
		\setlength{\columnsep}{-3.5cm}
		\begin{multicols}{2}
			\includegraphics[width=4cm]{message_passing.pdf}
			\columnbreak{}

			$$
			\int_{V_a\backslash X_u} f_{V_a | V_s}\left(\underline{x}_{V_a\backslash X_A} | V_s = \underline{y}_{V_s}\right)
			d\underline{x}_{V_a\backslash X_A} \propto
			$$
			$$
			\int_{\underline{X}_B,\underline{X}_D,\underline{X}_C} \psi_{U,A} \psi_{B,A} \psi_{C,A} \psi_{V,C} \psi_{D,C}d\underline{x}_Bd\underline{x}_Dd\underline{x}_C =
			$$
			\pause
			$$
			\psi_{U,A} \int \psi_{B,A}d\underline{x}_B \int \psi_{C,A} \psi_{V,C}
			\int \psi_{D,C}d\underline{x}_Dd\underline{x}_C=
			$$
		\end{multicols}
	\end{frame}

	\begin{frame}
		\begin{align*}
			\text{Za } u,v \in V_a\text{: } &
			\mu_{u \to v}\left(\underline{x}_v\right) \coloneqq
			\int \psi_{u,v}\left(\underline{x}_u,\underline{x}_v\right)
			\prod_{t\in N\left(u\right)\backslash v}
			\mu_{t \to u}\left(\underline{x}_u\right)d\underline{x}_u
			\\
			\text{Za }u \in V_s\text{ in }v \in V_a\text{: } &
			\mu_{u \to v}\left(\underline{x}_v\right) \coloneqq
			\psi_{u,v}\left(\underline{y}_u,\underline{x}_v\right)
		\end{align*}
		\pause
		$$
		\Rightarrow
		\psi_{U,A} \int \psi_{B,A}d\underline{x}_B \int \psi_{C,A} \psi_{V,C}
		\int \psi_{D,C}d\underline{x}_Dd\underline{x}_C=
		$$
		\pause
		$$
		\mu_{U \to A}\left(\underline{x}_A\right)\mu_{B \to A}\left(\underline{x}_A\right)
		\int \psi_{C,A}
		\mu_{V \to C}\left(\underline{x}_C\right)
		\mu_{D \to C}\left(\underline{x}_C\right)d\underline{x}_C=
		$$
		\pause
		$$
		\mu_{U \to A}\left(\underline{x}_A\right)\mu_{B \to A}\left(\underline{x}_A\right)
		\mu_{C \to A}\left(\underline{x}_A\right) =
		$$
		$$
		\prod_{u \in N\left(A\right)}\mu_{u \to A}\left(\underline{x}_A\right)
		$$
	\end{frame}

	\begin{frame}
		\frametitle{Za grafe s cikli}
		Naj bosta $u,v \in V_a$, $i \in \mathbb{N}$ pa število iteracije.
		$$
		M_u^{\left(i\right)}\left(\underline{x}_u\right) \coloneqq
		\prod_{t \in N\left(u\right)}\mu_{t \to u}\left(\underline{x}_u\right)
		$$
		$$
		\mu_{v \to u}^{\left(i\right)}\left(\underline{x}_u\right) \coloneqq
		\int \psi_{v,u}\left(\underline{x}_v,\underline{x}_u\right)
		\frac{
			M_v^{\left(i-1\right)}\left(\underline{x}_v\right)}{
			\mu_{u \to v}^{\left(i-1\right)}\left(\underline{x}_v\right)
		}d\underline{x}_v
		$$
		Za drevesa to konvergira k pravim vrednostim.
	\end{frame}

\end{document}

