\documentclass{beamer}
\usepackage[utf8]{inputenc}
\usepackage[slovene]{babel}
\usepackage{amsmath}
\usepackage{mathtools}
\usetheme{Madrid}
\setbeamertemplate{navigation symbols}{}

\title[Razširjanje upanja]{Razširjanje upanja za lokalizacijo senzorskih omrežji}
\author{Jaka Velkaverh}
\institute{UL FMF}
\date{18.12.2023}

\theoremstyle{definition}
\newtheorem{definicija}{Definicija}

\begin{document}

	\frame{\titlepage}

	\begin{frame}
		\frametitle{Opis problema}
		\begin{block}{Lokalizacija senzorskega omrežja}
			Omrežje si predstavljamo kot utežen graf, kjer so naprave vozlišča.
			Dve vozlišči sta povezani, če sta pripadajoči napravi dovolj blizu,
			da lahko izmerita razdaljo. Utež povezav je izmerjena razdalja,
			kjer pri merjenju pride do napak.

			Vemo lokacije od nekaterih naprav, tem rečemo sidra. Izračunati
			želimo približke za lokacije ostalih, tem rečemo agenti.
		\end{block}
	\end{frame}
	\begin{frame}
		\frametitle{Verjetnostni pogled}
		Oznake
		\begin{itemize}
			\item Vozlišča označujemo z črkami $t, u, s$.
			\item $D_{t,u} = d_{t,u} + \nu_{u, v},$ naključne spremenljivke razdalje.
			\item $\underline{X}_v$ naključne spremenljivke lokacij agentov.
		\end{itemize}

		Želimo izračunati gostoto porazdelitve
		$$f_v\left(\underline{x}_v | \left\{D_{t,u} \text{ take kot izmerjene}\right\}\right)$$
	\end{frame}

	\begin{frame}
		\frametitle{Markovska polja}
		\begin{block}{Pogojna neodvisnost}
			Nakjučna vektorja $\underline{X}, \underline{Y}$ sta pogojno
			neodvisna ob $\underline{Z}$, če velja:
			$$f_{\underline{X}, \underline{Y}}\left(\underline{x}, \underline{y}\ |\ \underline{Z} = \underline{z}\right) =
			  f_{\underline{X}}\left(\underline{x}\ |\ \underline{Z} = \underline{z}\right) \
			  f_{\underline{Y}}\left(\underline{y}\ |\ \underline{Z} = \underline{z}\right)
			$$
		\end{block}
		\begin{definicija}
			Graf $G = \left(V, E\right)$, in naključni slučajni vektorji
			$\underline{X}_v$, indeksirani z $v \in V$ tvorijo Markovsko polje,
			če velja Markova lastnost po parih:

			Za vsaka nesosedna $t, u \in V$ sta $\underline{X}_t$, $\underline{X}_u$
			pogojno neodvisna ob vseh ostalih
			$\underline{X}_{V\backslash\left\{t, u\right\}}$
		\end{definicija}
	\end{frame}

	\begin{block}{Hammersley–Cliffordov izrek}
		Naj bo $G = \left(V, E\right)$ in $\underline{X}_v$ kot prej in dodatno
		$f_{\underline{X}_V}\left(\underline{x}_V\right) > 0\quad \forall \underline{x}_V$.
		Tedaj $G$ in $\underline{X}_V$ tvorita Markovsko polje natanko tedaj,
		ko se skupna gostota faktorizira po klikah grafa $G$:
		$$f_{\underline{X}_V}\left(\underline{x}_V\right) =
			\prod_{C\;\text{klika}\;G}\psi_C\left(\underline{x}_C\right)$$
	\end{block}

\end{document}

